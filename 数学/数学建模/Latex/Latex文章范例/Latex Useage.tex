\documentclass[UTF8]{ctexart} % 添加中文支持

% documentclass到begin之间称为导言区,可以在这里进行一些全局设置

% 使用usepackage来添加宏包
% 所谓宏包,就是一系列控制序列的合集,这些控制序列太常用,以至于人们会觉得每次将他们写在导言区太过繁琐,于是将他们打包放在同一个文件中
% 宏包就是用于拓展Latex功能的
\usepackage{graphicx} % Required for inserting images
\usepackage{amsmath}

% latexindent用于tex文件格式化,删除多余空行只需添加-m参数即可

\title{文章标题}
\author{文章作者}
\date{\today}

\begin{document}

% 根据导言区设置生成标题、作者、日期
\maketitle % Insert the title, author and date

\newpage

\begin{abstract}
    摘要内容
\end{abstract}

\newpage

% 生成目录(需要注意的是,目录的正确生成至少需要编译两次)
\tableofcontents

\newpage

\section{一级标题}

\subsection{二级标题}

\subsubsection{三级标题}

\paragraph{段落标题}段落内容

\subparagraph{子段落标题}子段落内容

% *表示该章节不会出现在目录中(2,3级子标题同理)
\section*{一级标题(不在目录内)}

% 短标题:[]内的内容会出现在目录中,{}内的内容会出现在正文中,适用于标题过长无法在目录or页眉内完全显示的情况
\section[短标题]{This is a Very Long Title That Might Not Fit on One Line in the Table of Contents or the Header}

\end{document}
