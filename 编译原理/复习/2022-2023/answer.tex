\documentclass[UTF8]{ctexart} % 添加中文支持

% documentclass到begin之间称为导言区,可以在这里进行一些全局设置

% 使用usepackage来添加宏包
% 所谓宏包,就是一系列控制序列的合集,这些控制序列太常用,以至于人们会觉得每次将他们写在导言区太过繁琐,于是将他们打包放在同一个文件中
% 宏包就是用于拓展Latex功能的
\usepackage{graphicx} % 用于导入外部图片的宏包(推荐格式pdf>>>>png>jpg>eps)
\usepackage{amsmath} % 使用 AMS-LaTeX 提供的数学功能
\usepackage{lmodern} % 解决字体警告问题
% \usepackage[pdf]{graphviz} % graphviz绘图支持(需要安装graphviz)
% 我的评价是还不如把latex和graphviz分开使用(latex渲染,graphviz绘图,不必非得把两者合并到一起)
\usepackage{float} % 防止图片乱浮动导致图片文字顺序混乱的包
\usepackage{multirow} % 多行表格合并的宏包
\usepackage{diagbox} % 表头斜线分割宏包
\usepackage{listings} % 代码块宏包
\usepackage{color} % 颜色宏包
\usepackage{arydshln} % 表格虚线宏包

\title{2022-2023编译原理卷A}
\author{Garone Lombard}
\date{\today}

\begin{document}

% 根据导言区设置生成标题、作者、日期
\maketitle % Insert the title, author and date

\newpage

\begin{abstract}
    编译原理2022-2023试卷答案自行整理
\end{abstract}

\newpage

% 生成目录(需要注意的是,目录的正确生成至少需要编译两次)
\tableofcontents

\newpage

\section{填空}

\paragraph{1} 123

\section{正则文法与自动机}

\section{符号表构造与运行时存储分析}

\section{LL(1)分析法}

\section{算符优先分析法}

\section{SLR分析法}

\section{代码优化}

\end{document}